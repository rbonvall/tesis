\chapter{GPU implementation}
\label{ch:implementation}

This chapter describes how the simulation
was implemented on the GPU.

The reference algorithm for evaluating
this kind of computation on the GPU
is the one by Nyland et~al.\@~\cite[\S31]{gems3},
which shows how to integrate a body system
subject to gravitational interactions.
Some modifications to the algorithm
had to be made in order to make it applicable to Vortex Methods.

\section{Work distribution}
\label{sec:work-distribution}

The simulation of the Lamb-Oseen vortex
with the algorithm described in chapter~\ref{ch:vm-design}
amounts to the numerical integration
of the convection and diffusion equations
for each of the~\(N\) particles of the system:
\begin{align}
    \label{eq:x-integration}
    x_p^{(n + 1)} &= x_p^{(n)} + u_p^{(n)}\, δt, \\
    \label{eq:y-integration}
    y_p^{(n + 1)} &= y_p^{(n)} + v_p^{(n)}\, δt, \\
    \label{eq:a-integration}
    α_p^{(n + 1)} &= α_p^{(n)} + \dot{α}_p^{(n)}\, δt.
\end{align}

At every time step~\(n\),
and for every particle~\(p\),
the values of \(u_p\), \(v_p\) and \(\dot{α}_p\)
must be evaluated as a function of all other particles
using the Biot-Savart law and the Particle Strength Exchange formula:
\begin{align}
    %\label{eq:evaluation}
%    (u_p, v_p) &= \sum_q α_p \K_ε(\x_p - \x_q) \\
%    \dot{α}_p  &= νε^{-2} \sum_q [α_q - α_p]\,η_ε(\x_p - \x_q)
    \label{eq:bs-sum}
    (u_p, v_p) &= \sum_q \vec{U}_{pq},   & \text{where } \vec{U}_{pq} &= α_p \K_ε(\x_p - \x_q), \text{and} \\
    \label{eq:pse-sum}
    \dot{α}_p  &= νε^{-2} \sum_q A_{pq}, & \text{where }       A_{pq} &= [α_q - α_p]\,η_ε(\x_p - \x_q).
\end{align}
Summations~\eqref{eq:bs-sum} and~\eqref{eq:pse-sum} can be computed
independently among particles,
provided that the descriptions~\(x_q, y_q, α_q\)
for all the particles are available.

On the GPU, thus,
each thread is assigned the tasks
of computing sums~\eqref{eq:bs-sum} and~\eqref{eq:pse-sum} for one particle,
and of updating the particle description
according to equations~\eqref{eq:x-integration},
\eqref{eq:y-integration} and~\eqref{eq:a-integration}.

The~\(N\) threads are grouped into \(B\)~blocks of \(T = N/B\)~threads
(if~\(B\nmid N\), the last block will have some idle threads).
All threads within a block execute concurrently
(but not necessarily simultaneously),
while all blocks are scheduled independent of each other
to be executed when multiprocessors are available.

Thread~\(t\) in block~\(b\) evaluates \(\vec u_p\) and \(\dot{α}_p\)
for particle~\(p = bT + t\).
Conversely,
particle~\(p\) is evaluated by thread~\(t = p\bmod T\)
in block~\(b = \lfloor p/T \rfloor\).

\tikzstyle{innergrid}=[gray!80]
\tikzstyle{tile}=[green!60!black, thick]
\tikzstyle{sync}=[red!60!black, thick, dashed]
\tikzstyle{thread}=[->, decorate, decoration={snake, amplitude=.6}]
\tikzstyle{examplecell}=[fill=orange, opacity=.5]
\tikzstyle{kernelcall}=[#1, rounded corners=2pt, very thick, pattern color=#1]
\tikzstyle{kernelarrow}=[->, thick, out=90, in=180]
\tikzstyle{imglabel}=[anchor=west, text opacity=1.0, fill=white, fill opacity=.5,
                      text height=1ex, text depth=.25ex, rounded corners]

\newcommand\drawthread[4]{%
  \pgfmathsetmacro{\ybot}{(1 + \nrblockbodies) * #2}
  \draw [#1] (0.5, \ybot + 0.5 + #3) -- ++(#4 - 1, 0);%
}

\begin{figure}
  \centering
  \begin{tikzpicture}[scale=.2, yscale=-1, font=\small]

    \def\nrbodies{45}     \def\lastbody{44}
    \def\nrblocks{6}      \def\lastblock{5}
    \def\nrtiles{6}       \def\lasttile{5}
    \def\nrblockbodies{8} \def\lastblockbody{7}

    \foreach \block in {0,1,...,\lastblock} {
      %\def\ybot{9 * \block}
      \pgfmathsetmacro{\ybot}{(1 + \nrblockbodies) * \block}
      \draw[innergrid] (0, \ybot) grid      ++(\nrbodies, \nrblockbodies);
      \draw[tile]      (0, \ybot) rectangle ++(\nrbodies, \nrblockbodies);
      %\draw (1 + \nrbodies, \ybot + \nrblockbodies / 2)
      \draw (-11, \ybot + \nrblockbodies / 2)
          node[anchor=west] { block $\block$ };

      \foreach \syncpoint in {1,...,\lasttile}
        \draw[sync] (\nrblockbodies * \syncpoint, \ybot) -- ++(0, \nrblockbodies);
    }

    %\def\block{2}
    %\foreach \k in {0,...,3}
    %  \drawthread{thread}{\block}{\k}{43};
    %\foreach \k in {4,...,\lastblockbody}
    %  \drawthread{thread}{\block}{\k}{42};

    \def\block{1}
    \foreach \k in {0,1}
      \drawthread{thread}{\block}{\k}{14};
    \foreach \k in {2,...,\lastblockbody}
      \drawthread{thread}{\block}{\k}{11};

    \def\block{2}
    \foreach \k in {0,...,\lastblockbody}
      \drawthread{thread}{\block}{\k}{24.5};

    \def\block{4}
    \def\to{35}
    \foreach \k in {0,1,3,4,...,\lastblockbody}
      \drawthread{thread, gray}{\block}{\k}{\to};

    % draw kernel calls
    \def\currentx{34}
    \pgfmathsetmacro{\tilestart}{floor(\currentx / \nrblockbodies) * \nrblockbodies}
    \pgfmathsetmacro{\ybot}{(\nrblockbodies + 1) * \block}
    \def\currenty{\ybot + 2}
    \draw[kernelcall=brown, pattern=north east lines]
      (   0, \currenty - .2) rectangle ++(\nrbodies, 1.4);
    \draw[kernelcall=blue!50!black, pattern=north west lines]
      (\tilestart, \currenty - .1) rectangle ++(\nrblockbodies, 1.2);
    \draw[kernelcall=orange, fill]
      (\currentx, \currenty) rectangle ++( 1, 1);
    \node[anchor=base] (interactionanchor) at (\currentx + 0.5, \currenty + 1.0) {};
    \node[anchor=base] (tileanchor)        at (\currentx + 2.5, \currenty + 1.1) {};
    \node[anchor=base] (evalanchor)        at (\currentx - 8.5, \currenty + 1.2) {};

    \node[imglabel] (interactionlabel) at (49.5, \currenty +  4.5) {\verb!particle_interaction!};
    \node[imglabel] (tilelabel)        at (49.5, \currenty +  7.5) {\verb!update_tile!};
    \node[imglabel] (evallabel)        at (49.5, \currenty + 10.5) {\verb!eval_derivatives!};

    \draw[kernelarrow, orange]        (interactionanchor.base) to (interactionlabel);
    \draw[kernelarrow, blue!50!black] (tileanchor.base)        to (tilelabel);
    \draw[kernelarrow, brown]         (evalanchor.base)        to (evallabel);

    \drawthread{thread}{\block}{2}{\to};

    \def\examplex{21}
    \def\exampley{5}
    \draw[examplecell] (\examplex, \exampley) rectangle ++(1, 1);
    \node[anchor=base] (cellanchor) at (\examplex + .5, \exampley + .5) {};

    \draw (35, -4) node[imglabel, anchor=west] (interaction) {
      \(
       \left\{
       \begin{aligned}
         {\vec U}_{pq} &= α_q\,\K_ε(\x_p - \x_q) \\
         A_{pq}        &= [α_q - α_p]\,η_ε(\x_p - \x_q) \\
       \end{aligned}
       \right.
      \)
    };
    \draw[->, out=-45, in=180] (cellanchor.base) to (interaction);
    \node[imglabel, anchor=east]  (plabel) at (\examplex - 7,  \exampley + .5) {\((x_p, y_p, α_p)\)};
    \node[imglabel, anchor=south] (qlabel) at (\examplex + .5, -1) {\((x_q, y_q, α_q)\)};
    \draw[dotted, very thick] (plabel) -- (cellanchor.base);
    \draw[dotted, very thick] (qlabel) -- (cellanchor.base);

    \foreach \block in {1,2,4} {
      \pgfmathsetmacro{\ybot}{(1 + \nrblockbodies) * \block}
      \foreach \thread in {0,...,\lastblockbody} {
        \node[font=\tiny, anchor=east] at (0, .5 + \ybot + \thread) {\thread};
      }
    }

    \def\nridlethreads{3}
    \fill[pattern=north east lines] (0, 50) rectangle ++(\nrbodies, 3);
    \node[imglabel, anchor=center] (idlelabel) at (\nrbodies / 2, 51.5) {idle threads};

  \end{tikzpicture}
  \caption[Work distribution among GPU threads.]%
    {Work distribution among threads and thread blocks
    for a system of \protect{\(N = 45\)} particles and
    a grid of \protect{\(B = 6\)} blocks of \(T = 8\) threads.
    Each small cell represents an interaction between two particles.
    Waved lines show how threads sequentially compute
    and add up interactions along a row.
    Vertical dashed lines show where threads synchronize
    and load a new set of particles into shared memory.
    The boxes in block 4 show the particle interactions associated
    to each of the device kernels being called.
  }
  \label{fig:nbody-tiles}
\end{figure}

Figure~\ref{fig:nbody-tiles} shows a schematic representation
of the distribution of work among threads.
All of the~\(N^2\) particle interactions are represented
as cells in the finer grid.
Each thread computes interactions sequentially along a row.
In this example, blocks 1 through 4 are already in execution,
while the others have not started yet.

\section{Memory management}
\label{sec:memory-management}

Throughout the simulation,
the current state of the particle system
is stored in an array of particles in global memory,
where each particle is described as a 3-tuple~\((x_p, y_p, α_p)\).

On Nvidia GPUs,
global memory reads are always~\emph{coalesced} (accessed in chunks)
and \emph{aligned} to chunk boundaries~\cite{cudaprog2}.
Unaligned accesses impose a performance penalty,
since more chunks must be fetched
when data spills across boundaries.

Since chunks are aligned to addresses that are
multiples of 32, 64 or 128 (depending of the chunk size),
accesses are more efficient when particles are stored
as values of type~\texttt{float4} instead of~\texttt{float3}
(records of resp.\@ four and three single-precision floating point values)
whose fourth field is kept unused.

Each thread within block \(b\)
has a local variable with the description of the particle \(p\)
for which it will compute \(\vel_p\) and \(\dot{α}_p\).

Each block of threads has access to its own shared memory,
that is significantly faster than global memory
(two orders of magnitude), but also much smaller.
In order for the particle descriptions to fit into shared memory,
they have to be fetched in smaller blocks.
The algorithm remains simpler if these blocks are of size \(T\),
since then each thread can fetch one particle, and the global effect
is that the block of threads loads the block of particles at once.
These memory loads must be interleaved with barrier synchronizations,
because some threads can be ahead of others, and could start computing
on particles that the latter are assigned to fetch but have not yet.

The group of particle interactions computed
with a given block of particles that are in shared memory
will be refered as a \emph{computational tile}.
In figure~\ref{fig:nbody-tiles},
tiles are delimited by dashed vertical lines
representing synchronization and data transfer points.

\section{Kernel implementation}
\label{sec:kernel-implementation}

The global kernel that computes one iteration of the simulation
is called \verb!integrate!.
This kernel is invoked from the host CPU,
and before its execution
the array of particles needs to already be copied
to the GPU's global memory.

The global kernel is accompanied by several device kernels,
which are invoked from CUDA code, and serve mainly
for organizing the code into smaller units.

The pseudocode for the \verb!integrate! kernel is the following:
\begin{lstlisting}
GLOBAL KERNEL integrate(dt, N, old_particles, new_particles)
    p = id of particle associated to current thread
    IF p < N THEN
        particle = old_particles[p]
        derivatives = eval_derivatives(particle, old_particles, N)
        particle = particle + derivatives * dt
        new_particles[p] = particle
    END IF
END KERNEL
\end{lstlisting}

This kernel is executed by each of the \(T\cdot B\) threads,
and basically it fetches one particle from global memory,
invokes the evaluation of the derivatives at its position,
updates its value and puts it back into global memory.
The global effect is that all particles are integrated
concurrently.

Two global arrays are used: one for reading particle descriptions
computed at the previous iteration (\verb!old_particles!)
and one to put the updated particle descriptions (\verb!new_particles!).
After each iteration, these arrays must be swapped.

The actual evaluation is started by the device kernel
\verb!eval_derivatives!, that is described
by the following pseudocode:

\begin{lstlisting}
DEVICE KERNEL eval_derivatives(current_particle, particles, N)
    shared_particles = shared array of particles
    t = current thread id
    num_tiles = ceiling(N / T)

    derivatives = (0, 0, 0)
    FOR tile = 0 TO num_tiles - 1 DO
        shared_particles[t] = particles[tile * T + t]
        synchronize()
        derivatives = update_tile(particle, derivatives)
        synchronize()
    END FOR

    RETURN derivatives
END KERNEL
\end{lstlisting}

\verb!eval_derivatives! is called once for each particle.
At the block level,
its effect is that
blocks of particles are loaded into shared memory
and a computational tile is executed,
and this is done iteratively until all the interactions are evaluated
for the particles associated to its threads.

This \emph{read-synchronize-compute-synchronize} loop
is a very common pattern in GPU programming
when the algorithm operates on data in a blockwise fashion.

The \verb!update_tile! device kernel
computes one tile
with the data that is already loaded into shared memory:
\begin{lstlisting}
DEVICE KERNEL update_tile(particle, derivatives)
    shared_particles = shared array of particles
    t = current thread id

    i = 0
    WHILE counter < block_size
        derivatives = particle_interaction(derivatives, particle,
                                 shared_particles[i + block_size * t])
        i = i + 1
    END WHILE

    derivatives.z = derivatives.z * nu/e^2
    RETURN derivatives
END KERNEL
\end{lstlisting}

Finally, the computation of the interaction
between particles \(p\) and \(q\)
is done by the \verb!particle_interaction! device kernel:

\begin{lstlisting}
DEVICE KERNEL particle_interaction(derivatives, p, q)
    r = p - q
    velocity_kernel_factor = k_factor(r^2)
    derivatives.x += velocity_kernel_factor * -r.y
    derivatives.y += velocity_kernel_factor *  r.x
    derivatives.z += (q.z - p.z) * eta(r^2)
END KERNEL
\end{lstlisting}
where \verb!k_factor! and \verb!eta! are device kernels
evaluating the scalar part of \(\K_ε\) and \(η_ε\),
respectively.

