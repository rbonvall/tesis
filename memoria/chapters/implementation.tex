\chapter{GPU implementation}
\label{ch:implementation}

This chapter describes how the simulation was implemented on the GPU.

\section{Work distribution}
\label{sec:work-distribution}

The simulation of the Lamb-Oseen vortex
with the algorithm described in chapter~\ref{ch:vm-design}
amounts to the numerical integration
of the convection and diffusion equations
for each of the particles of the system:
\begin{align}
    %\label{eq:integration}
    x_p^{(n + 1)} &= x_p^{(n)} + u_p^{(n)}\, δt, \\
    y_p^{(n + 1)} &= y_p^{(n)} + v_p^{(n)}\, δt, \\
    α_p^{(n + 1)} &= α_p^{(n)} + \dot{α}_p^{(n)}\, δt.
\end{align}

At every time step~\(n\), the values of \(u\), \(v\) and \(\dot{α}\)
must be evaluated as a function of all other particles:
\begin{align}
    %\label{eq:evaluation}
    (u_p, v_p) &= \sum_q α_p \K_ε(\x_p - \x_q) \\
    \dot{α}_p  &= νε^{-2} \sum_q (α_q - α_p)\,η_ε(\x_p - \x_q)
\end{align}
The inner terms of both summations have to be computed
for every pair of particles.

The reference algorithm for evaluating
this kind of computation on the GPU
is the one by Nyland et.~al~\cite[\S31]{gems3},
which shows how to integrate a body system under gravitational interaction.




