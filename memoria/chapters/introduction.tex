\chapter{Introduction}
\label{ch:introduction}

Fluid dynamics is the study of fluids in motion.

While being a discipline of great theoretical interest in itself,
fluid dynamics has many industrial applications:
fluid flow problems arise in aeronautics, meteorology,
astrophysics, and oceanography, among many other disciplines.

The equations governing fluid motion have been known
for a couple of centuries.
The most important ones are the Navier-Stokes equations,
which describe the law of conservation of momentum
for an element of volume in a viscous fluid:
\begin{equation}
  ρ(\vel_t + \vel\cdot∇\vel) = \text{forces per volume unit},
\end{equation}
where \(\vel\) is the flow velocity and
\(ρ\) is the fluid density.

Under some conditions,
the solution of this equation can exhibit chaotic behavior
called turbulence.
Turbulent flows are highly irregular
and extremely difficult to solve.
The main contribution to turbulence
is given by the  \(ρ\vel\cdot∇\vel\) term
(called convective acceleration),
that describes the effect that the geometry of the flow
has on the fluid at a fixed time.
The nonlinearity of this term
is what makes the equations difficult to solve.

The Navier-Stokes equations are usually supplemented
with the mass and energy conservation laws
and with equations of state from thermodynamics,
in order to fully describe the flow.
The need for these additional equations
depends on which assumptions have been made about the flow.

Since all but the simplest flow problems
cannot be solved analytically,
numerical methods are indispensable
for practical applications.
Computational fluid dynamics (CFD)
is the branch of fluid dynamics
that studies efficient, robust and reliable algorithms
for solving the underlying partial differential equations.

Vortex Methods are one kind of CFD methods,
that are based on the discretization
of the vorticity field using particles as computational elements,
whose evolution is determined
by the Lagrangian description of the Navier-Stokes equations.
In the Lagrangian formulation,
the convective acceleration term dissapears,
so Vortex Methods bypass some of the numerical problems
that need to be addressed explicitly in Eulerian methods.

Despite Vortex Methods having been formulated several decades ago,
their applicability for actual engineering problems was hindered
because of the computational cost of evaluating the velocity field
and their inability to solve viscous effects accurately.
These difficulties have been addressed in several ways since,
so they have become a practical alternative
for solving incompressible flows with high Reynolds number.

In general,
fluid flow simulations are among
the most computationally demanding numerical problems.
As is the case
for all High-Performance Computing applications nowadays,
effective use of parallelism is crucial for
simulating fluid phenomena within a reasonable time.
In order to be practical,
large supercomputers or computer clusters
are usually needed for simulating the flow accurately.

One of the parallel architectures
that is currently being used as a target
for running this kind of applications
is the Graphics Processing Unit (GPU).
Originally designed to be a fixed-function processor
for accelerating stages of the graphics pipeline,
the architecture of the GPU has evolved
into a fully programmable array of multiprocessor,
capable of executing hundreds of threads simultaneously.
GPUs have been increasingly used over the last half-decade
to accelerate general-purpose expensive computations
because of their affordable cost and ubiquity.
By exploiting massive parallelism,
GPUs can achieve up to two orders of magnitud
better performance than a CPU
for certain applications.

Programming for the GPU is different from doing it
for ordinary processors.
Each task has to be organized into a grid of blocks,
that are distributed among multiprocessors.
One block has many threads that must execute
exactly the same instructions.
There are several types of memory,
each optimized for different kinds of usages
and with different access speeds.
Also, the developer must be aware of
a lot more of details of the hardware,
and this is specially true
when programming for maximum performance.

In this undergraduate thesis work,
a parallel 2-D Vortex Method is designed
for simulating the Lamb-Oseen vortex,
and implemented in the CUDA programming environmentt.
The algorithm runs entirely on the GPU,
and simulates both the convection and the diffusion of the flow.

In chapter~\ref{ch:vm},
the theory of Vortex Methods for two-dimensional
incompressible flows is introduced,
along with their most important algorithmic features.
In chapter~\ref{ch:gpu-computing},
the architecture of the GPU is explained,
and the programming model
of CUDA is described.
Chapter~\ref{ch:vm-design}
introduces the Lamb-Oseen vortex,
and explains the design decisions
for the Vortex Method to be implemented.
Chapter~\ref{ch:implementation}
discusses the GPU implementation,
the task distribution and the CUDA kernels.
Chapter~\ref{ch:experiments}
describes the experiments performed
and presents the results.

