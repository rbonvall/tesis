\chapter{Vortex Methods}
\label{ch:vm}

Vortex Methods are a class of methods
used for direct numerical simulations
of incompressible viscous flows at high Reynold numbers~\cite{cottet00}.
In this chapter, the 2-D Vortex Blob Method is introduced,
along with its algorithmic characteristics.

\section{The mathematics of vorticity}
\label{sec:eqs-vort}

The vorticity field \(\mathbold\vort\)
of a fluid flow with velocity field~\(\vel = (u, v, w)\)
is defined as:%
\begin{equation}
  \mathbold\vort = \nabla\times\vel.
\end{equation}

Incompressible fluids are those for which
the volume of fluid elements remains constant in time.
This condition is expressed mathematically as~\(\nabla\cdot\vel = 0\).
It follows from the definition of vorticity that also~\(\nabla\cdot\mathbold\vort = 0\).

The vorticity is orthogonal to the velocity on every point of the fluid.
It follows that in a two-dimensional flow,
the only non-zero component of the vorticity is the \(z\) component. 
It this case, it is customary to consider the vorticity as a scalar field
\(\vort = v_x - u_y\)\footnotemark,
such that \(\mathbold\vort = \vort\hat{\vec k}\).

\footnotetext{Unless otherwise stated, subindices represent partial derivatives.}

The motion of an incompressible fluid
is governed by the conservations laws of mass and momentum.
These laws are typically presented in an Eulerian reference frame,
where the equations are developed from the local analysis of the flow
in a fixed location in space, and are respectively:
\begin{align}
  \rho_t + \divergence(\rho\vel) &= 0, \\
  \rho(\vel_t + \vel\cdot\nabla\vel) &= -\nabla p + \mu\lapl\vel;
\end{align}
where \(\rho\) and \(p\) are the density and the pressure fields,
and \(\mu\) is the dynamic viscosity of the fluid.

Vortex Methods, on the other hand, are based on the Lagrangian reference frame,
which views the fluid as a collection of fluid elements
that are freely translating, rotating and deforming,
while carrying the dependent quantities of the flow field,
such as velocity and temperature.
A full description of the flow is obtained by
identifying the initial location of the fluid elements
and the initial value of the dependent variable.

Both reference frames are related by the material derivative operator \(D/Dt\),
defined as:
\begin{equation}
  \frac{D}{Dt} = \frac{\partial}{\partial t} + (\vel\cdot\nabla).
\end{equation}



\section{The 2-D Vortex Blob Method}
\label{sec:vortex-blob-method}

A two-dimensional incompressible unbounded flow


Vorticity-velocity formulation of the Navier-Stokes equations:

\begin{equation}
  \label{eq:vort-vel-navier-stokes}
  \frac{\partial\omega}{\partial t} + \divergence(\vel\vort) = \lapl\vort
\end{equation}

\begin{align}
  \vort(\,\cdot\,{}, 0) &= \vort_0, \\
  \divergence\vel &= 0, \\
  \curl\vel &= \vort, \\
  \norm\vel &\to\vel_\infty
\end{align}

Biot-Savart law

\begin{equation}
  \label{eq:biot-savart-law}
  \vel = \vel_\infty + \K * \vort
\end{equation}
where \(\K = \nabla\times G\),
\(G\) being the Green's function for the Laplacian.
In \(\R^2\):

\begin{align}
  G(\x) &= -(2\pi)^{-1} \log\abs{\vec x} \\
  \K(\x) &= (2\pi\abs{x}^2)^{-1} \bigl(-x_2, x_1\bigr) 
\end{align}

\begin{equation}
  \label{eq:initial-vorticity-approximation}
  \vort_0\approx\vort_0^h = \sum_p\circulation_p\delta(\x - \x_p)
\end{equation}

\begin{equation}
  \label{eq:point-vortex-method-approximation}
  \vort^h(\x, t) = \sum_p\circulation_p\delta\bigl(\x - \x_p^h(t)\bigr)
\end{equation}


\begin{align}
  \label{eq:vortex-blob-ode}
  \frac{d\x_p^h}{dt} &= \vel(\vec x_p^t, t) \\
  \x_p^h(0) &= \x_p \\
  \vel &= \K_\e * \vort
\end{align}

\begin{equation}
  \zeta_\e(\x) = \e^{-2}\zeta(\x/\e)
\end{equation}

\begin{equation}
  \K_\e = \K * \zeta_\e
\end{equation}

\begin{equation}
  \vort_\e^h = \sum_p\zeta_\e(\x - \x_p^h)
\end{equation}

