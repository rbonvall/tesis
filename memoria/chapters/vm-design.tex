\chapter{Design of a Vortex Method for the Lamb-Oseen vortex}
\label{ch:vm-design}


This chapter introduces the fluid problem to be simulated,
and the design decisions %...

\section{The Lamb-Oseen vortex}
\label{sec:lamb-oseen-vortex}

The Lamb-Oseen vortex is the simplest viscous vortex.
It is a two-dimensional flow with circular symmetry
whose vorticity decays over time due to viscosity.

Its analytical form is:
\begin{equation}
  \label{eq:lamb-oseen-vorticity}
  ω(r, t) = \frac{Γ_0}{4πνt} \exp(-r^2/4νt),
\end{equation}
where \(Γ_0\) is the total circulation contained in the vortex
\(ν\) is the viscosity of the fluid.
and \(r^2 = x^2 + y^2\).
The tangential velocity is given by:
\begin{equation}
  \label{eq:lamb-oseen-tangential-velocity}
  u_\theta(r, t) = \frac{Γ_0}{2πr} \bigl(1 - \exp(-r^2/4νt)\bigr),
\end{equation}
which translates into the velocity vector
\(\vec{u} = (-y, x)\cdot u_\theta(r, t)/r \).

The Reynolds number of the vortex
can be defined as \(\text{Re} = Γ_0/ν\).

Because it has a known analytical solution,
the Lamb-Oseen vortex is typically used
for testing fluid flow simulations.

\section{Discretization and initialization}
\label{sec:lamb-oseen-discretization}

The simulation for the Lamb-Oseen vortex
needs to start at a time other than zero,
because of the~\(t\) factor in the denominators
in equation~\eqref{eq:lamb-oseen-vorticity}.
\(t_0 = 4.0\) was chosen arbitrarily a starting time.




